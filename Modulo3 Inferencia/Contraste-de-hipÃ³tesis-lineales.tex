% Options for packages loaded elsewhere
\PassOptionsToPackage{unicode}{hyperref}
\PassOptionsToPackage{hyphens}{url}
%
\documentclass[
]{article}
\usepackage{amsmath,amssymb}
\usepackage{iftex}
\ifPDFTeX
  \usepackage[T1]{fontenc}
  \usepackage[utf8]{inputenc}
  \usepackage{textcomp} % provide euro and other symbols
\else % if luatex or xetex
  \usepackage{unicode-math} % this also loads fontspec
  \defaultfontfeatures{Scale=MatchLowercase}
  \defaultfontfeatures[\rmfamily]{Ligatures=TeX,Scale=1}
\fi
\usepackage{lmodern}
\ifPDFTeX\else
  % xetex/luatex font selection
\fi
% Use upquote if available, for straight quotes in verbatim environments
\IfFileExists{upquote.sty}{\usepackage{upquote}}{}
\IfFileExists{microtype.sty}{% use microtype if available
  \usepackage[]{microtype}
  \UseMicrotypeSet[protrusion]{basicmath} % disable protrusion for tt fonts
}{}
\makeatletter
\@ifundefined{KOMAClassName}{% if non-KOMA class
  \IfFileExists{parskip.sty}{%
    \usepackage{parskip}
  }{% else
    \setlength{\parindent}{0pt}
    \setlength{\parskip}{6pt plus 2pt minus 1pt}}
}{% if KOMA class
  \KOMAoptions{parskip=half}}
\makeatother
\usepackage{xcolor}
\usepackage[margin=1in]{geometry}
\usepackage{color}
\usepackage{fancyvrb}
\newcommand{\VerbBar}{|}
\newcommand{\VERB}{\Verb[commandchars=\\\{\}]}
\DefineVerbatimEnvironment{Highlighting}{Verbatim}{commandchars=\\\{\}}
% Add ',fontsize=\small' for more characters per line
\usepackage{framed}
\definecolor{shadecolor}{RGB}{248,248,248}
\newenvironment{Shaded}{\begin{snugshade}}{\end{snugshade}}
\newcommand{\AlertTok}[1]{\textcolor[rgb]{0.94,0.16,0.16}{#1}}
\newcommand{\AnnotationTok}[1]{\textcolor[rgb]{0.56,0.35,0.01}{\textbf{\textit{#1}}}}
\newcommand{\AttributeTok}[1]{\textcolor[rgb]{0.13,0.29,0.53}{#1}}
\newcommand{\BaseNTok}[1]{\textcolor[rgb]{0.00,0.00,0.81}{#1}}
\newcommand{\BuiltInTok}[1]{#1}
\newcommand{\CharTok}[1]{\textcolor[rgb]{0.31,0.60,0.02}{#1}}
\newcommand{\CommentTok}[1]{\textcolor[rgb]{0.56,0.35,0.01}{\textit{#1}}}
\newcommand{\CommentVarTok}[1]{\textcolor[rgb]{0.56,0.35,0.01}{\textbf{\textit{#1}}}}
\newcommand{\ConstantTok}[1]{\textcolor[rgb]{0.56,0.35,0.01}{#1}}
\newcommand{\ControlFlowTok}[1]{\textcolor[rgb]{0.13,0.29,0.53}{\textbf{#1}}}
\newcommand{\DataTypeTok}[1]{\textcolor[rgb]{0.13,0.29,0.53}{#1}}
\newcommand{\DecValTok}[1]{\textcolor[rgb]{0.00,0.00,0.81}{#1}}
\newcommand{\DocumentationTok}[1]{\textcolor[rgb]{0.56,0.35,0.01}{\textbf{\textit{#1}}}}
\newcommand{\ErrorTok}[1]{\textcolor[rgb]{0.64,0.00,0.00}{\textbf{#1}}}
\newcommand{\ExtensionTok}[1]{#1}
\newcommand{\FloatTok}[1]{\textcolor[rgb]{0.00,0.00,0.81}{#1}}
\newcommand{\FunctionTok}[1]{\textcolor[rgb]{0.13,0.29,0.53}{\textbf{#1}}}
\newcommand{\ImportTok}[1]{#1}
\newcommand{\InformationTok}[1]{\textcolor[rgb]{0.56,0.35,0.01}{\textbf{\textit{#1}}}}
\newcommand{\KeywordTok}[1]{\textcolor[rgb]{0.13,0.29,0.53}{\textbf{#1}}}
\newcommand{\NormalTok}[1]{#1}
\newcommand{\OperatorTok}[1]{\textcolor[rgb]{0.81,0.36,0.00}{\textbf{#1}}}
\newcommand{\OtherTok}[1]{\textcolor[rgb]{0.56,0.35,0.01}{#1}}
\newcommand{\PreprocessorTok}[1]{\textcolor[rgb]{0.56,0.35,0.01}{\textit{#1}}}
\newcommand{\RegionMarkerTok}[1]{#1}
\newcommand{\SpecialCharTok}[1]{\textcolor[rgb]{0.81,0.36,0.00}{\textbf{#1}}}
\newcommand{\SpecialStringTok}[1]{\textcolor[rgb]{0.31,0.60,0.02}{#1}}
\newcommand{\StringTok}[1]{\textcolor[rgb]{0.31,0.60,0.02}{#1}}
\newcommand{\VariableTok}[1]{\textcolor[rgb]{0.00,0.00,0.00}{#1}}
\newcommand{\VerbatimStringTok}[1]{\textcolor[rgb]{0.31,0.60,0.02}{#1}}
\newcommand{\WarningTok}[1]{\textcolor[rgb]{0.56,0.35,0.01}{\textbf{\textit{#1}}}}
\usepackage{graphicx}
\makeatletter
\def\maxwidth{\ifdim\Gin@nat@width>\linewidth\linewidth\else\Gin@nat@width\fi}
\def\maxheight{\ifdim\Gin@nat@height>\textheight\textheight\else\Gin@nat@height\fi}
\makeatother
% Scale images if necessary, so that they will not overflow the page
% margins by default, and it is still possible to overwrite the defaults
% using explicit options in \includegraphics[width, height, ...]{}
\setkeys{Gin}{width=\maxwidth,height=\maxheight,keepaspectratio}
% Set default figure placement to htbp
\makeatletter
\def\fps@figure{htbp}
\makeatother
\setlength{\emergencystretch}{3em} % prevent overfull lines
\providecommand{\tightlist}{%
  \setlength{\itemsep}{0pt}\setlength{\parskip}{0pt}}
\setcounter{secnumdepth}{-\maxdimen} % remove section numbering
\ifLuaTeX
  \usepackage{selnolig}  % disable illegal ligatures
\fi
\IfFileExists{bookmark.sty}{\usepackage{bookmark}}{\usepackage{hyperref}}
\IfFileExists{xurl.sty}{\usepackage{xurl}}{} % add URL line breaks if available
\urlstyle{same}
\hypersetup{
  pdftitle={Módulo3},
  pdfauthor={María Sánchez Paniagua},
  hidelinks,
  pdfcreator={LaTeX via pandoc}}

\title{Módulo3}
\author{María Sánchez Paniagua}
\date{2024-03-29}

\begin{document}
\maketitle

Vamos a ver dos ejemplos para hipótesis lineales con una sola función
paramétrica estimable, para un sólo parámetro (test T). El tercer
ejemplo será para varias (test F).

\hypertarget{ejemplo1-regresiuxf3n-muxfaltiple-rango-muxe1ximo-h0-bi-0}{%
\section{Ejemplo1: Regresión múltiple (RANGO MÁXIMO) (H0: Bi =
0)}\label{ejemplo1-regresiuxf3n-muxfaltiple-rango-muxe1ximo-h0-bi-0}}

Por ejemplo, vamos a contrastar si el coeficiente de regresión de la
variable Area es cero.

\begin{Shaded}
\begin{Highlighting}[]
\FunctionTok{data}\NormalTok{(gala, }\AttributeTok{package=}\StringTok{"faraway"}\NormalTok{)}
\NormalTok{lmod }\OtherTok{\textless{}{-}} \FunctionTok{lm}\NormalTok{(Species }\SpecialCharTok{\textasciitilde{}}\NormalTok{ Area }\SpecialCharTok{+}\NormalTok{ Elevation }\SpecialCharTok{+}\NormalTok{ Nearest }\SpecialCharTok{+}\NormalTok{ Scruz }\SpecialCharTok{+}\NormalTok{ Adjacent,}
\AttributeTok{data =}\NormalTok{ gala)}
\end{Highlighting}
\end{Shaded}

\hypertarget{la-matriz-de-diseuxf1o}{%
\subsection{La matriz de diseño}\label{la-matriz-de-diseuxf1o}}

\begin{Shaded}
\begin{Highlighting}[]
\NormalTok{X }\OtherTok{\textless{}{-}} \FunctionTok{model.matrix}\NormalTok{(lmod)}
\NormalTok{QR }\OtherTok{\textless{}{-}} \FunctionTok{qr}\NormalTok{(X)}
\NormalTok{QR}\SpecialCharTok{$}\NormalTok{rank }\CommentTok{\# Esde rango máximo}
\end{Highlighting}
\end{Shaded}

\begin{verbatim}
## [1] 6
\end{verbatim}

\begin{Shaded}
\begin{Highlighting}[]
\NormalTok{n }\OtherTok{\textless{}{-}} \FunctionTok{dim}\NormalTok{(X)[}\DecValTok{1}\NormalTok{]}
\NormalTok{r }\OtherTok{\textless{}{-}} \FunctionTok{dim}\NormalTok{(X)[}\DecValTok{2}\NormalTok{] }
\end{Highlighting}
\end{Shaded}

\hypertarget{comprobamos-que-el-summary-es-igual-a-los-cuxe1lculos-manuales}{%
\subsection{Comprobamos que el summary es igual a los cálculos
manuales}\label{comprobamos-que-el-summary-es-igual-a-los-cuxe1lculos-manuales}}

\begin{Shaded}
\begin{Highlighting}[]
\NormalTok{sum.lmod }\OtherTok{\textless{}{-}} \FunctionTok{summary}\NormalTok{(lmod)}
\NormalTok{ee.Area }\OtherTok{\textless{}{-}}\NormalTok{ sum.lmod}\SpecialCharTok{$}\NormalTok{sigma }\SpecialCharTok{*} \FunctionTok{sqrt}\NormalTok{(sum.lmod}\SpecialCharTok{$}\NormalTok{cov.unscaled[}\DecValTok{2}\NormalTok{,}\DecValTok{2}\NormalTok{])}
\NormalTok{t.est }\OtherTok{\textless{}{-}} \FunctionTok{coef}\NormalTok{(lmod)[}\DecValTok{2}\NormalTok{] }\SpecialCharTok{/}\NormalTok{ ee.Area}
\end{Highlighting}
\end{Shaded}

El summary es:

\begin{Shaded}
\begin{Highlighting}[]
\NormalTok{sum.lmod}
\end{Highlighting}
\end{Shaded}

\begin{verbatim}
## 
## Call:
## lm(formula = Species ~ Area + Elevation + Nearest + Scruz + Adjacent, 
##     data = gala)
## 
## Residuals:
##      Min       1Q   Median       3Q      Max 
## -111.679  -34.898   -7.862   33.460  182.584 
## 
## Coefficients:
##              Estimate Std. Error t value Pr(>|t|)    
## (Intercept)  7.068221  19.154198   0.369 0.715351    
## Area        -0.023938   0.022422  -1.068 0.296318    
## Elevation    0.319465   0.053663   5.953 3.82e-06 ***
## Nearest      0.009144   1.054136   0.009 0.993151    
## Scruz       -0.240524   0.215402  -1.117 0.275208    
## Adjacent    -0.074805   0.017700  -4.226 0.000297 ***
## ---
## Signif. codes:  0 '***' 0.001 '**' 0.01 '*' 0.05 '.' 0.1 ' ' 1
## 
## Residual standard error: 60.98 on 24 degrees of freedom
## Multiple R-squared:  0.7658, Adjusted R-squared:  0.7171 
## F-statistic:  15.7 on 5 and 24 DF,  p-value: 6.838e-07
\end{verbatim}

La primera es el estimador del modelo lineal, la sugunda el error
estándar, la tercera es dividir el estimador por el error estandar y R
nos ofrece el cáclulo del p-valor.

\hypertarget{estimate}{%
\paragraph{Estimate}\label{estimate}}

Compruebo que el e estimate calculado de forma manual da lo mismo que en
el summary

\begin{Shaded}
\begin{Highlighting}[]
\NormalTok{(beta.Area }\OtherTok{\textless{}{-}} \FunctionTok{coef}\NormalTok{(lmod)[}\DecValTok{2}\NormalTok{]) }\CommentTok{\# Devuelve el valor de estimate del summary}
\end{Highlighting}
\end{Shaded}

\begin{verbatim}
##        Area 
## -0.02393834
\end{verbatim}

\hypertarget{error-estuxe1ndar}{%
\paragraph{Error estándar}\label{error-estuxe1ndar}}

Compruebo que el errore estándar calculado de forma manual da lo mismo
que en el summary

\begin{Shaded}
\begin{Highlighting}[]
\NormalTok{a }\OtherTok{\textless{}{-}} \FunctionTok{c}\NormalTok{(}\DecValTok{0}\NormalTok{,}\DecValTok{1}\NormalTok{,}\DecValTok{0}\NormalTok{,}\DecValTok{0}\NormalTok{,}\DecValTok{0}\NormalTok{,}\DecValTok{0}\NormalTok{) }\CommentTok{\# Vector columna para el ee (solo quiero el segundo elemento de los coef, el beta)}
\CommentTok{\# Calculo el error estándar (ee)}
\NormalTok{(ee.beta.Area }\OtherTok{\textless{}{-}} \FunctionTok{sqrt}\NormalTok{(sum.lmod}\SpecialCharTok{$}\NormalTok{sigma}\SpecialCharTok{\^{}}\DecValTok{2} \SpecialCharTok{*} \FunctionTok{t}\NormalTok{(a) }\SpecialCharTok{\%*\%} \FunctionTok{solve}\NormalTok{(}\FunctionTok{crossprod}\NormalTok{(X)) }\SpecialCharTok{\%*\%}\NormalTok{ a))}
\end{Highlighting}
\end{Shaded}

\begin{verbatim}
##            [,1]
## [1,] 0.02242235
\end{verbatim}

\hypertarget{t-valor}{%
\paragraph{T-valor}\label{t-valor}}

Podemos comprobar que este valor es el que figura en la tercera columna
del summary(lmod), cociente de la primera y segunda columnas.

\begin{Shaded}
\begin{Highlighting}[]
\NormalTok{t.est}
\end{Highlighting}
\end{Shaded}

\begin{verbatim}
##      Area 
## -1.067611
\end{verbatim}

\begin{Shaded}
\begin{Highlighting}[]
\NormalTok{sum.lmod}\SpecialCharTok{$}\NormalTok{coef[}\DecValTok{2}\NormalTok{,}\DecValTok{3}\NormalTok{]}
\end{Highlighting}
\end{Shaded}

\begin{verbatim}
## [1] -1.067611
\end{verbatim}

\begin{Shaded}
\begin{Highlighting}[]
\NormalTok{t.est2 }\OtherTok{\textless{}{-}}\NormalTok{ beta.Area }\SpecialCharTok{/}\NormalTok{ ee.beta.Area}
\NormalTok{t.est2}
\end{Highlighting}
\end{Shaded}

\begin{verbatim}
##           [,1]
## [1,] -1.067611
\end{verbatim}

\hypertarget{el-p-valor}{%
\paragraph{El p-valor}\label{el-p-valor}}

El p-valor es de un \emph{contraste parcial} (teniendo en cuenta el
resto de variables, no es una decisión firme)

\begin{Shaded}
\begin{Highlighting}[]
\FunctionTok{pt}\NormalTok{(}\FunctionTok{abs}\NormalTok{(t.est), }\AttributeTok{df =} \DecValTok{30{-}6}\NormalTok{, }\AttributeTok{lower.tail =} \ConstantTok{FALSE}\NormalTok{) }\SpecialCharTok{*} \DecValTok{2}
\end{Highlighting}
\end{Shaded}

\begin{verbatim}
##     Area 
## 0.296318
\end{verbatim}

En el caso del p-valor por ejemplo de Area como es mayor de 0,05
aceptamos que el parametro del área es 0. (Teniendo en cceunta que e sun
contraste parcial).

\hypertarget{intervalos-de-confianza}{%
\paragraph{Intervalos de confianza}\label{intervalos-de-confianza}}

Se puede calcular a mano o con una función de R:

\begin{Shaded}
\begin{Highlighting}[]
\NormalTok{prob }\OtherTok{\textless{}{-}} \FunctionTok{c}\NormalTok{(}\FloatTok{0.05}\SpecialCharTok{/}\DecValTok{2}\NormalTok{, }\DecValTok{1}\FloatTok{{-}0.05}\SpecialCharTok{/}\DecValTok{2}\NormalTok{)}
\FunctionTok{coef}\NormalTok{(lmod)[}\DecValTok{2}\NormalTok{] }\SpecialCharTok{+} \FunctionTok{qt}\NormalTok{(prob, }\AttributeTok{df=}\DecValTok{30{-}6}\NormalTok{) }\SpecialCharTok{*}\NormalTok{ ee.Area}
\end{Highlighting}
\end{Shaded}

\begin{verbatim}
## [1] -0.07021580  0.02233912
\end{verbatim}

Aunque es mucho más sencillo utilizar la función confint().

\begin{Shaded}
\begin{Highlighting}[]
\FunctionTok{confint}\NormalTok{(lmod)[}\DecValTok{2}\NormalTok{,] }\CommentTok{\# Tomo el segundo que es el del área}
\end{Highlighting}
\end{Shaded}

\begin{verbatim}
##       2.5 %      97.5 % 
## -0.07021580  0.02233912
\end{verbatim}

\hypertarget{ejemplo2-diseuxf1o-cross-over-simplificado-h0-alfa-beta-no-rango-muxe1ximo}{%
\section{Ejemplo2: Diseño cross-over simplificado (H0: alfa = beta) (NO
RANGO
MÁXIMO)}\label{ejemplo2-diseuxf1o-cross-over-simplificado-h0-alfa-beta-no-rango-muxe1ximo}}

\begin{Shaded}
\begin{Highlighting}[]
\NormalTok{y }\OtherTok{\textless{}{-}} \FunctionTok{c}\NormalTok{(}\DecValTok{17}\NormalTok{,}\DecValTok{34}\NormalTok{,}\DecValTok{26}\NormalTok{,}\DecValTok{10}\NormalTok{,}\DecValTok{19}\NormalTok{,}\DecValTok{17}\NormalTok{,}\DecValTok{8}\NormalTok{,}\DecValTok{16}\NormalTok{,}\DecValTok{13}\NormalTok{,}\DecValTok{11}\NormalTok{,}
       \DecValTok{17}\NormalTok{,}\DecValTok{41}\NormalTok{,}\DecValTok{26}\NormalTok{,}\DecValTok{3}\NormalTok{,}\SpecialCharTok{{-}}\DecValTok{6}\NormalTok{,}\SpecialCharTok{{-}}\DecValTok{4}\NormalTok{,}\DecValTok{11}\NormalTok{,}\DecValTok{16}\NormalTok{,}\DecValTok{16}\NormalTok{,}\DecValTok{4}\NormalTok{,}
       \DecValTok{21}\NormalTok{,}\DecValTok{20}\NormalTok{,}\DecValTok{11}\NormalTok{,}\DecValTok{26}\NormalTok{,}\DecValTok{42}\NormalTok{,}\DecValTok{28}\NormalTok{,}\DecValTok{3}\NormalTok{,}\DecValTok{3}\NormalTok{,}\DecValTok{16}\NormalTok{,}\SpecialCharTok{{-}}\DecValTok{10}\NormalTok{,}
       \DecValTok{10}\NormalTok{,}\DecValTok{24}\NormalTok{,}\DecValTok{32}\NormalTok{,}\DecValTok{26}\NormalTok{,}\DecValTok{52}\NormalTok{,}\DecValTok{28}\NormalTok{,}\DecValTok{27}\NormalTok{,}\DecValTok{28}\NormalTok{,}\DecValTok{21}\NormalTok{,}\DecValTok{42}\NormalTok{)}

\NormalTok{h }\OtherTok{\textless{}{-}} \FunctionTok{length}\NormalTok{(y)}
\CommentTok{\# Cuatro columnas de la matriz de diseño}
\NormalTok{mu }\OtherTok{\textless{}{-}} \FunctionTok{rep}\NormalTok{(}\DecValTok{1}\NormalTok{,}\DecValTok{40}\NormalTok{)}
\NormalTok{alpha }\OtherTok{\textless{}{-}} \FunctionTok{c}\NormalTok{(}\FunctionTok{rep}\NormalTok{(}\DecValTok{1}\NormalTok{,}\DecValTok{10}\NormalTok{),}\FunctionTok{rep}\NormalTok{(}\DecValTok{0}\NormalTok{,}\DecValTok{10}\NormalTok{),}\FunctionTok{rep}\NormalTok{(}\DecValTok{0}\NormalTok{,}\DecValTok{10}\NormalTok{),}\FunctionTok{rep}\NormalTok{(}\DecValTok{1}\NormalTok{,}\DecValTok{10}\NormalTok{))}
\NormalTok{beta }\OtherTok{\textless{}{-}} \FunctionTok{c}\NormalTok{(}\FunctionTok{rep}\NormalTok{(}\DecValTok{0}\NormalTok{,}\DecValTok{10}\NormalTok{),}\FunctionTok{rep}\NormalTok{(}\DecValTok{1}\NormalTok{,}\DecValTok{10}\NormalTok{),}\FunctionTok{rep}\NormalTok{(}\DecValTok{1}\NormalTok{,}\DecValTok{10}\NormalTok{),}\FunctionTok{rep}\NormalTok{(}\DecValTok{0}\NormalTok{,}\DecValTok{10}\NormalTok{))}
\NormalTok{gamma }\OtherTok{\textless{}{-}} \FunctionTok{c}\NormalTok{(}\FunctionTok{rep}\NormalTok{(}\DecValTok{0}\NormalTok{,}\DecValTok{10}\NormalTok{),}\FunctionTok{rep}\NormalTok{(}\DecValTok{1}\NormalTok{,}\DecValTok{10}\NormalTok{),}\FunctionTok{rep}\NormalTok{(}\DecValTok{0}\NormalTok{,}\DecValTok{10}\NormalTok{),}\FunctionTok{rep}\NormalTok{(}\DecValTok{1}\NormalTok{,}\DecValTok{10}\NormalTok{))}
\end{Highlighting}
\end{Shaded}

\hypertarget{resoluciuxf3n-teuxf3rica-modelo-3}{%
\subsection{Resolución teórica (modelo
3)}\label{resoluciuxf3n-teuxf3rica-modelo-3}}

Para contrastar la hipótesis H0 : α − β = 0 del modelo crossover debemos
recuperar los datos del módulo anterior y algunos de sus elementos.

\begin{Shaded}
\begin{Highlighting}[]
\FunctionTok{library}\NormalTok{(MASS)}
\end{Highlighting}
\end{Shaded}

\begin{verbatim}
## Warning: package 'MASS' was built under R version 4.3.2
\end{verbatim}

\begin{Shaded}
\begin{Highlighting}[]
\NormalTok{cmod1 }\OtherTok{\textless{}{-}} \FunctionTok{lm}\NormalTok{(y }\SpecialCharTok{\textasciitilde{}}\NormalTok{ alpha }\SpecialCharTok{+}\NormalTok{ beta }\SpecialCharTok{+}\NormalTok{ gamma)}
\NormalTok{ss }\OtherTok{\textless{}{-}} \FunctionTok{summary}\NormalTok{(cmod1)}
\NormalTok{X.co }\OtherTok{\textless{}{-}} \FunctionTok{model.matrix}\NormalTok{(cmod1)}
\NormalTok{XtXginv }\OtherTok{\textless{}{-}} \FunctionTok{ginv}\NormalTok{(}\FunctionTok{t}\NormalTok{(X.co) }\SpecialCharTok{\%*\%}\NormalTok{ X.co)}
\NormalTok{coef.co }\OtherTok{\textless{}{-}}\NormalTok{ XtXginv }\SpecialCharTok{\%*\%} \FunctionTok{t}\NormalTok{(X.co) }\SpecialCharTok{\%*\%}\NormalTok{ y}
\NormalTok{a }\OtherTok{\textless{}{-}} \FunctionTok{c}\NormalTok{(}\DecValTok{0}\NormalTok{,}\DecValTok{1}\NormalTok{,}\SpecialCharTok{{-}}\DecValTok{1}\NormalTok{,}\DecValTok{0}\NormalTok{)}
\NormalTok{ee.a }\OtherTok{\textless{}{-}}\NormalTok{ sum.lmod}\SpecialCharTok{$}\NormalTok{sigma }\SpecialCharTok{*} \FunctionTok{sqrt}\NormalTok{(}\FunctionTok{t}\NormalTok{(a) }\SpecialCharTok{\%*\%}\NormalTok{ XtXginv }\SpecialCharTok{\%*\%}\NormalTok{ a)}
\NormalTok{t.est }\OtherTok{\textless{}{-}} \FunctionTok{sum}\NormalTok{(a}\SpecialCharTok{*}\NormalTok{coef.co) }\SpecialCharTok{/}\NormalTok{ ee.a}
\NormalTok{t.est}
\end{Highlighting}
\end{Shaded}

\begin{verbatim}
##           [,1]
## [1,] 0.4589762
\end{verbatim}

\begin{Shaded}
\begin{Highlighting}[]
\FunctionTok{pt}\NormalTok{(}\FunctionTok{abs}\NormalTok{(t.est), }\AttributeTok{df=}\DecValTok{40{-}3}\NormalTok{, }\AttributeTok{lower.tail =} \ConstantTok{FALSE}\NormalTok{) }\SpecialCharTok{*} \DecValTok{2}
\end{Highlighting}
\end{Shaded}

\begin{verbatim}
##           [,1]
## [1,] 0.6489363
\end{verbatim}

\begin{Shaded}
\begin{Highlighting}[]
\NormalTok{prob }\OtherTok{\textless{}{-}} \FunctionTok{c}\NormalTok{(}\FloatTok{0.05}\SpecialCharTok{/}\DecValTok{2}\NormalTok{, }\DecValTok{1}\FloatTok{{-}0.05}\SpecialCharTok{/}\DecValTok{2}\NormalTok{)}
\FunctionTok{sum}\NormalTok{(a}\SpecialCharTok{*}\NormalTok{coef.co) }\SpecialCharTok{+} \FunctionTok{qt}\NormalTok{(prob, }\AttributeTok{df=}\DecValTok{40{-}3}\NormalTok{) }\SpecialCharTok{*} \FunctionTok{as.vector}\NormalTok{(ee.a)}
\end{Highlighting}
\end{Shaded}

\begin{verbatim}
## [1] -30.21914  47.91914
\end{verbatim}

Luego rechazamos la hipótesis nula y admitimos la diferencia entre los
efectos de los fármacos \#\# Resolución Carmona

Como alfa menos beta es paramétrica estimable para este modelo sí se
puede estimar aunque no sea de rango máximo

\begin{Shaded}
\begin{Highlighting}[]
\CommentTok{\# Reúno la columnas en la matriz de diseño}
\NormalTok{X }\OtherTok{\textless{}{-}} \FunctionTok{matrix}\NormalTok{(}\FunctionTok{c}\NormalTok{(mu, alpha, beta, gamma), }\AttributeTok{ncol=}\DecValTok{4}\NormalTok{)}

\CommentTok{\# rango = 3 =\textgreater{} matriz de diseño sin rango máximo}
\NormalTok{r }\OtherTok{\textless{}{-}} \FunctionTok{qr}\NormalTok{(X)}\SpecialCharTok{$}\NormalTok{rank}
\end{Highlighting}
\end{Shaded}

Como los parámetros no tiene solución única, vamos a usar la g-inversa

\begin{Shaded}
\begin{Highlighting}[]
\CommentTok{\# Solución con g{-}inversa}
\FunctionTok{library}\NormalTok{(MASS)}
\NormalTok{XtX }\OtherTok{\textless{}{-}} \FunctionTok{crossprod}\NormalTok{(X) }\CommentTok{\# X crtapuesta de X}
\NormalTok{XtXinv }\OtherTok{\textless{}{-}} \FunctionTok{ginv}\NormalTok{(XtX) }\CommentTok{\# G inversa}
\CommentTok{\# Cálculo de parámetros (POSIBLE solución)}
\NormalTok{param }\OtherTok{\textless{}{-}}\NormalTok{ XtXinv }\SpecialCharTok{\%*\%} \FunctionTok{crossprod}\NormalTok{(X, y)}
\NormalTok{param}
\end{Highlighting}
\end{Shaded}

\begin{verbatim}
##           [,1]
## [1,] 11.033333
## [2,]  9.941667
## [3,]  1.091667
## [4,]  4.150000
\end{verbatim}

Con estos parametros podemos calcular la estimacion de la funcion
parametrica estimable:

\begin{Shaded}
\begin{Highlighting}[]
\CommentTok{\# Función paramétrica alpha{-}beta}
\NormalTok{a }\OtherTok{\textless{}{-}} \FunctionTok{c}\NormalTok{(}\DecValTok{0}\NormalTok{, }\DecValTok{1}\NormalTok{, }\SpecialCharTok{{-}}\DecValTok{1}\NormalTok{, }\DecValTok{0}\NormalTok{)}
\CommentTok{\# El estadístico sum(a * param)}
\NormalTok{est }\OtherTok{\textless{}{-}} \FunctionTok{t}\NormalTok{(a) }\SpecialCharTok{\%*\%}\NormalTok{ param }\CommentTok{\#}
\end{Highlighting}
\end{Shaded}

\hypertarget{estimacion-de-sigma-cuadrado}{%
\paragraph{Estimacion de sigma
cuadrado}\label{estimacion-de-sigma-cuadrado}}

\begin{Shaded}
\begin{Highlighting}[]
\CommentTok{\# MSE (estimador de sigma\^{}2), se hace a mano, no hay summary}
\NormalTok{residuos }\OtherTok{\textless{}{-}}\NormalTok{ y }\SpecialCharTok{{-}}\NormalTok{ X }\SpecialCharTok{\%*\%}\NormalTok{ param}
\NormalTok{MSE }\OtherTok{\textless{}{-}} \FunctionTok{sum}\NormalTok{(residuos}\SpecialCharTok{\^{}}\DecValTok{2}\NormalTok{)}\SpecialCharTok{/}\NormalTok{(n}\SpecialCharTok{{-}}\NormalTok{r) }\CommentTok{\# El mean square error, la estimación de sigma cuadrado}
\end{Highlighting}
\end{Shaded}

\hypertarget{error-estuxe1ndar-1}{%
\paragraph{Error estándar}\label{error-estuxe1ndar-1}}

\begin{Shaded}
\begin{Highlighting}[]
\CommentTok{\# error estándar de la estimación}
\NormalTok{ee.est }\OtherTok{\textless{}{-}} \FunctionTok{sqrt}\NormalTok{(MSE }\SpecialCharTok{*} \FunctionTok{t}\NormalTok{(a) }\SpecialCharTok{\%*\%}\NormalTok{ XtXinv }\SpecialCharTok{\%*\%}\NormalTok{ a)}
\NormalTok{ee.est}
\end{Highlighting}
\end{Shaded}

\begin{verbatim}
##          [,1]
## [1,] 4.771802
\end{verbatim}

\#\#\#\#t de Student

\begin{Shaded}
\begin{Highlighting}[]
\NormalTok{t.est }\OtherTok{\textless{}{-}}\NormalTok{ est}\SpecialCharTok{/}\NormalTok{ee.est}
\NormalTok{t.est}
\end{Highlighting}
\end{Shaded}

\begin{verbatim}
##          [,1]
## [1,] 1.854645
\end{verbatim}

\hypertarget{p-valor}{%
\paragraph{p valor}\label{p-valor}}

\begin{Shaded}
\begin{Highlighting}[]
\NormalTok{p\_valor }\OtherTok{\textless{}{-}} \FunctionTok{pt}\NormalTok{(}\FunctionTok{abs}\NormalTok{(t.est), }\AttributeTok{df =}\NormalTok{ n}\SpecialCharTok{{-}}\NormalTok{r, }\AttributeTok{lower.tail =} \ConstantTok{FALSE}\NormalTok{) }\SpecialCharTok{*} \DecValTok{2}
\NormalTok{p\_valor}
\end{Highlighting}
\end{Shaded}

\begin{verbatim}
##            [,1]
## [1,] 0.07459912
\end{verbatim}

p-valor \textless{} 0.05 ==\textgreater{} Rechazamos la H0 Luego hay
diferencias entre los fármacos, su efeccto es distinto

\hypertarget{intervalo-de-confianza1-alpha-para-alfa-menos-beta}{%
\paragraph{Intervalo de confianza(1-alpha) para alfa menos
beta}\label{intervalo-de-confianza1-alpha-para-alfa-menos-beta}}

Se hace a mano porque no hay función de R. Calculo el cuantil de la t de
student àra probabilidad de 1-00,5 entre dos

\begin{Shaded}
\begin{Highlighting}[]
\NormalTok{t.alpha }\OtherTok{\textless{}{-}} \FunctionTok{qt}\NormalTok{(}\DecValTok{1}\FloatTok{{-}0.05}\SpecialCharTok{/}\DecValTok{2}\NormalTok{, }\AttributeTok{df =}\NormalTok{ n}\SpecialCharTok{{-}}\NormalTok{r)}
\NormalTok{IC }\OtherTok{\textless{}{-}} \FunctionTok{c}\NormalTok{(est }\SpecialCharTok{{-}}\NormalTok{ t.alpha }\SpecialCharTok{*}\NormalTok{ ee.est, est }\SpecialCharTok{+}\NormalTok{ t.alpha }\SpecialCharTok{*}\NormalTok{ ee.est)}
\NormalTok{IC}
\end{Highlighting}
\end{Shaded}

\begin{verbatim}
## [1] -0.9409286 18.6409286
\end{verbatim}

\hypertarget{ejemplo3-contraste-de-modelos-con-f-de-fisher}{%
\section{Ejemplo3: Contraste de modelos con F de
Fisher}\label{ejemplo3-contraste-de-modelos-con-f-de-fisher}}

En el caso de varias funciones paramétricas al mismo tiempo, no se usa
la t de student para contrastarlas por separado (contraste múltiple,
provoca problemas en los errores), queremos hacer un constraste global.

Este contraste de un conjunto de hipotesis lienales utilizamos tambien
una notacion matricial, de forma que cada fila de A es una de las
hipótesis, una funcion paramétrica estimable FPE.

El rango q de la matriz de hipótesis A (nº de hipótesis) es menor que el
de la matriz de diseño

Se utilizará el \emph{test F de Fisher}, una generalizacion del t de
Student (porque si q es 1, solo hay una FPE es la t de Student elevada
al cuadrado)

H1: Y = Xbeta + e (rango X = r) H0: Y = Xbeta + e, Abeta = 0 (rango A =
q)

Lo que se hace en la hipótesis nula con Abeta = 0 es una restrición en
los parámetros. Y esta restriccion transforma los parametros beta y la
matriz de diseño X en un \emph{Modelo lineal de la hipótesis nula}.

Para esto usamos la funcion \emph{anova(hipot\_nula, hipot\_general)}

\hypertarget{ejemplo-video-galuxe1pagos}{%
\subsubsection{Ejemplo video
(Galápagos)}\label{ejemplo-video-galuxe1pagos}}

\hypertarget{test-de-significacion-de-regresiuxf3n}{%
\paragraph{Test de significacion de
regresión}\label{test-de-significacion-de-regresiuxf3n}}

Hacemos un test de significacion de regresion.

Veo si todos los coeficientes de la regresion son cero excepto el
termino independiente (beta0) o mu.

Esto es un contraste sobre varios coeficientes por lo que no aplica la t
de Student.

El modelo de hipotesis general es:

\begin{Shaded}
\begin{Highlighting}[]
\FunctionTok{data}\NormalTok{(gala, }\AttributeTok{package=}\StringTok{"faraway"}\NormalTok{)}
\NormalTok{lmod }\OtherTok{\textless{}{-}} \FunctionTok{lm}\NormalTok{(Species }\SpecialCharTok{\textasciitilde{}}\NormalTok{ Area }\SpecialCharTok{+}\NormalTok{ Elevation }\SpecialCharTok{+}\NormalTok{ Nearest }\SpecialCharTok{+}\NormalTok{ Scruz }\SpecialCharTok{+}\NormalTok{ Adjacent,}
\AttributeTok{data =}\NormalTok{ gala)}
\end{Highlighting}
\end{Shaded}

La hipótesis nula es

\begin{Shaded}
\begin{Highlighting}[]
\CommentTok{\# Hipótesis nula: Todos los coeficientes son cero (excepto beta0)}
\NormalTok{nullmod }\OtherTok{\textless{}{-}} \FunctionTok{lm}\NormalTok{(Species }\SpecialCharTok{\textasciitilde{}} \DecValTok{1}\NormalTok{, }\AttributeTok{data =}\NormalTok{ gala) }\CommentTok{\# El 1 indica que solo tome el intersect como variable constante}
\end{Highlighting}
\end{Shaded}

Comparo los modelos con el test F:

\begin{Shaded}
\begin{Highlighting}[]
\CommentTok{\# Test F}
\FunctionTok{anova}\NormalTok{(nullmod, lmod)}
\end{Highlighting}
\end{Shaded}

\begin{verbatim}
## Analysis of Variance Table
## 
## Model 1: Species ~ 1
## Model 2: Species ~ Area + Elevation + Nearest + Scruz + Adjacent
##   Res.Df    RSS Df Sum of Sq      F    Pr(>F)    
## 1     29 381081                                  
## 2     24  89231  5    291850 15.699 6.838e-07 ***
## ---
## Signif. codes:  0 '***' 0.001 '**' 0.01 '*' 0.05 '.' 0.1 ' ' 1
\end{verbatim}

Aparece la suma de cuadrado de los dos modelos, la F con los gardos de
libertad y el p-valor que es muy pequeño. Se rechaza la hipótesis nula
(todos los coeficientes son 0 excepto intercept).

Es bueno rechazar esta hipotesis porque si no la regresion es inutil

\begin{Shaded}
\begin{Highlighting}[]
\FunctionTok{summary}\NormalTok{(lmod)}
\end{Highlighting}
\end{Shaded}

\begin{verbatim}
## 
## Call:
## lm(formula = Species ~ Area + Elevation + Nearest + Scruz + Adjacent, 
##     data = gala)
## 
## Residuals:
##      Min       1Q   Median       3Q      Max 
## -111.679  -34.898   -7.862   33.460  182.584 
## 
## Coefficients:
##              Estimate Std. Error t value Pr(>|t|)    
## (Intercept)  7.068221  19.154198   0.369 0.715351    
## Area        -0.023938   0.022422  -1.068 0.296318    
## Elevation    0.319465   0.053663   5.953 3.82e-06 ***
## Nearest      0.009144   1.054136   0.009 0.993151    
## Scruz       -0.240524   0.215402  -1.117 0.275208    
## Adjacent    -0.074805   0.017700  -4.226 0.000297 ***
## ---
## Signif. codes:  0 '***' 0.001 '**' 0.01 '*' 0.05 '.' 0.1 ' ' 1
## 
## Residual standard error: 60.98 on 24 degrees of freedom
## Multiple R-squared:  0.7658, Adjusted R-squared:  0.7171 
## F-statistic:  15.7 on 5 and 24 DF,  p-value: 6.838e-07
\end{verbatim}

En la ultima linea del summary ya nos lo decia. \emph{Es necesario para
dar significacion a la regresion.}

\hypertarget{test-h0-betaarea-betaadjacent}{%
\subsubsection{Test H0: betaArea =
betaAdjacent}\label{test-h0-betaarea-betaadjacent}}

Miro si el coeficiente de area (bets) es igual al de adjacent.

Si la h0 es cierta, si los dos parametros son iguales podriamos sumar la
variable area y adjacent y darle a esa suma un unico parametro, Hay un
parametro pero es el mismo. Para eso está la función \emph{I()}.

La funcion I() dice que primero sume y luego el resultado es una
variable

Es una sola funcion parametrica estimable por lo que puedo usar el t de
Student, pero seeria sobre beta de area menos beta de adjacent y no es
inmedianto, es mas sencillo el F.

\begin{Shaded}
\begin{Highlighting}[]
\CommentTok{\# Hipotesis nula}
\NormalTok{lmod0 }\OtherTok{\textless{}{-}} \FunctionTok{lm}\NormalTok{(Species }\SpecialCharTok{\textasciitilde{}} \FunctionTok{I}\NormalTok{(Area }\SpecialCharTok{+}\NormalTok{ Adjacent) }\SpecialCharTok{+}\NormalTok{ Elevation }\SpecialCharTok{+}\NormalTok{ Nearest }\SpecialCharTok{+}\NormalTok{ Scruz, }\AttributeTok{data =}\NormalTok{ gala)}

\CommentTok{\# Test F}
\FunctionTok{anova}\NormalTok{(lmod0, lmod)}
\end{Highlighting}
\end{Shaded}

\begin{verbatim}
## Analysis of Variance Table
## 
## Model 1: Species ~ I(Area + Adjacent) + Elevation + Nearest + Scruz
## Model 2: Species ~ Area + Elevation + Nearest + Scruz + Adjacent
##   Res.Df    RSS Df Sum of Sq     F  Pr(>F)  
## 1     25 109591                             
## 2     24  89231  1     20360 5.476 0.02793 *
## ---
## Signif. codes:  0 '***' 0.001 '**' 0.01 '*' 0.05 '.' 0.1 ' ' 1
\end{verbatim}

Hay un grado de libertad (q) y 24 (30-6) el estadistico F es 5,47 y el
p-valor es 0,027. Rechazamos que ambos parametros sean iguales.

\hypertarget{ejemplo-libro-modulo-3-pdf}{%
\subsubsection{Ejemplo libro (modulo 3
pdf)}\label{ejemplo-libro-modulo-3-pdf}}

\begin{Shaded}
\begin{Highlighting}[]
\NormalTok{lmod }\OtherTok{\textless{}{-}} \FunctionTok{lm}\NormalTok{(Species }\SpecialCharTok{\textasciitilde{}}\NormalTok{ ., }\AttributeTok{data =}\NormalTok{ gala[,}\SpecialCharTok{{-}}\DecValTok{2}\NormalTok{])}
\NormalTok{lmod0 }\OtherTok{\textless{}{-}} \FunctionTok{lm}\NormalTok{(Species }\SpecialCharTok{\textasciitilde{}} \DecValTok{1}\NormalTok{, }\AttributeTok{data =}\NormalTok{ gala[,}\SpecialCharTok{{-}}\DecValTok{2}\NormalTok{])}
\FunctionTok{anova}\NormalTok{(lmod0,lmod)}
\end{Highlighting}
\end{Shaded}

\begin{verbatim}
## Analysis of Variance Table
## 
## Model 1: Species ~ 1
## Model 2: Species ~ Area + Elevation + Nearest + Scruz + Adjacent
##   Res.Df    RSS Df Sum of Sq      F    Pr(>F)    
## 1     29 381081                                  
## 2     24  89231  5    291850 15.699 6.838e-07 ***
## ---
## Signif. codes:  0 '***' 0.001 '**' 0.01 '*' 0.05 '.' 0.1 ' ' 1
\end{verbatim}

En el caso de que el contraste de modelos se haga con una única
restricción, es decir q = 1, el test F es equivalente al test t, ya que
F = t2.

Por ejemplo, el contraste de la hipótesis βArea = 0 se puede resolver
con un contraste de modelos.

\begin{Shaded}
\begin{Highlighting}[]
\NormalTok{lmod }\OtherTok{\textless{}{-}} \FunctionTok{lm}\NormalTok{(Species }\SpecialCharTok{\textasciitilde{}}\NormalTok{ ., }\AttributeTok{data=}\NormalTok{gala[,}\SpecialCharTok{{-}}\DecValTok{2}\NormalTok{])}
\NormalTok{lmod0 }\OtherTok{\textless{}{-}} \FunctionTok{lm}\NormalTok{(Species }\SpecialCharTok{\textasciitilde{}}\NormalTok{ Elevation }\SpecialCharTok{+}\NormalTok{ Nearest }\SpecialCharTok{+}\NormalTok{ Scruz }\SpecialCharTok{+}\NormalTok{ Adjacent,}
\AttributeTok{data =}\NormalTok{ gala[,}\SpecialCharTok{{-}}\DecValTok{2}\NormalTok{])}
\FunctionTok{anova}\NormalTok{(lmod0,lmod)}
\end{Highlighting}
\end{Shaded}

\begin{verbatim}
## Analysis of Variance Table
## 
## Model 1: Species ~ Elevation + Nearest + Scruz + Adjacent
## Model 2: Species ~ Area + Elevation + Nearest + Scruz + Adjacent
##   Res.Df   RSS Df Sum of Sq      F Pr(>F)
## 1     25 93469                           
## 2     24 89231  1    4237.7 1.1398 0.2963
\end{verbatim}

Observemos que el p-valor es el mismo que obtuvimos con el estadístico t
y además t2 = F.

\begin{Shaded}
\begin{Highlighting}[]
\NormalTok{ss }\OtherTok{\textless{}{-}} \FunctionTok{summary}\NormalTok{(lmod)}
\NormalTok{sum.lmod}\SpecialCharTok{$}\NormalTok{coef[}\DecValTok{2}\NormalTok{,}\DecValTok{4}\NormalTok{] }\CommentTok{\# p{-}valor}
\end{Highlighting}
\end{Shaded}

\begin{verbatim}
## [1] 0.296318
\end{verbatim}

\begin{Shaded}
\begin{Highlighting}[]
\NormalTok{sum.lmod}\SpecialCharTok{$}\NormalTok{coef[}\DecValTok{2}\NormalTok{,}\DecValTok{3}\NormalTok{]}\SpecialCharTok{\^{}}\DecValTok{2} \CommentTok{\# t\^{}2}
\end{Highlighting}
\end{Shaded}

\begin{verbatim}
## [1] 1.139792
\end{verbatim}

Otra hipótesis que se puede resolver como un contraste de modelos es H0
:βArea = βAdjacent. Si los dos coeficientes son iguales, podemos
considerar que son uno solo y sumar las dos variables.

\begin{Shaded}
\begin{Highlighting}[]
\NormalTok{lmod0 }\OtherTok{\textless{}{-}} \FunctionTok{lm}\NormalTok{(Species }\SpecialCharTok{\textasciitilde{}} \FunctionTok{I}\NormalTok{(Area }\SpecialCharTok{+}\NormalTok{ Adjacent) }\SpecialCharTok{+}\NormalTok{ Elevation }\SpecialCharTok{+}\NormalTok{ Nearest }\SpecialCharTok{+}
\NormalTok{Scruz, }\AttributeTok{data =}\NormalTok{ gala[,}\SpecialCharTok{{-}}\DecValTok{2}\NormalTok{])}
\FunctionTok{anova}\NormalTok{(lmod0, lmod)}
\end{Highlighting}
\end{Shaded}

\begin{verbatim}
## Analysis of Variance Table
## 
## Model 1: Species ~ I(Area + Adjacent) + Elevation + Nearest + Scruz
## Model 2: Species ~ Area + Elevation + Nearest + Scruz + Adjacent
##   Res.Df    RSS Df Sum of Sq     F  Pr(>F)  
## 1     25 109591                             
## 2     24  89231  1     20360 5.476 0.02793 *
## ---
## Signif. codes:  0 '***' 0.001 '**' 0.01 '*' 0.05 '.' 0.1 ' ' 1
\end{verbatim}

En este caso, rechazamos la hipótesis considerada.

Una hipótesis del tipo H0 : βElevation = 0.5 también se puede contrastar
así

\begin{Shaded}
\begin{Highlighting}[]
\NormalTok{lmod0 }\OtherTok{\textless{}{-}} \FunctionTok{lm}\NormalTok{(Species }\SpecialCharTok{\textasciitilde{}}\NormalTok{ Area }\SpecialCharTok{+} \FunctionTok{offset}\NormalTok{(}\FloatTok{0.5} \SpecialCharTok{*}\NormalTok{ Elevation) }\SpecialCharTok{+}\NormalTok{ Nearest }\SpecialCharTok{+}
\NormalTok{Scruz }\SpecialCharTok{+}\NormalTok{ Adjacent, }\AttributeTok{data =}\NormalTok{ gala[,}\SpecialCharTok{{-}}\DecValTok{2}\NormalTok{])}
\FunctionTok{anova}\NormalTok{(lmod0, lmod)}
\end{Highlighting}
\end{Shaded}

\begin{verbatim}
## Analysis of Variance Table
## 
## Model 1: Species ~ Area + offset(0.5 * Elevation) + Nearest + Scruz + 
##     Adjacent
## Model 2: Species ~ Area + Elevation + Nearest + Scruz + Adjacent
##   Res.Df    RSS Df Sum of Sq      F   Pr(>F)   
## 1     25 131312                                
## 2     24  89231  1     42081 11.318 0.002574 **
## ---
## Signif. codes:  0 '***' 0.001 '**' 0.01 '*' 0.05 '.' 0.1 ' ' 1
\end{verbatim}

En este caso también rechazamos la hipótesis considerada.

\hypertarget{ejemplo-crossover-modulo-3-pdf}{%
\subsection{Ejemplo crossover (modulo 3
pdf)}\label{ejemplo-crossover-modulo-3-pdf}}

En el diseño crossover la principal hipótesis H0 : α = β se puede
contrastar con un test F. Observemos que si los dos efectos son iguales,
el parámetro común es el mismo en las cuatro situaciones experimentales
y se confunde con la media general μ.

\begin{Shaded}
\begin{Highlighting}[]
\NormalTok{y }\OtherTok{\textless{}{-}} \FunctionTok{c}\NormalTok{(}\DecValTok{17}\NormalTok{,}\DecValTok{34}\NormalTok{,}\DecValTok{26}\NormalTok{,}\DecValTok{10}\NormalTok{,}\DecValTok{19}\NormalTok{,}\DecValTok{17}\NormalTok{,}\DecValTok{8}\NormalTok{,}\DecValTok{16}\NormalTok{,}\DecValTok{13}\NormalTok{,}\DecValTok{11}\NormalTok{,}
       \DecValTok{17}\NormalTok{,}\DecValTok{41}\NormalTok{,}\DecValTok{26}\NormalTok{,}\DecValTok{3}\NormalTok{,}\SpecialCharTok{{-}}\DecValTok{6}\NormalTok{,}\SpecialCharTok{{-}}\DecValTok{4}\NormalTok{,}\DecValTok{11}\NormalTok{,}\DecValTok{16}\NormalTok{,}\DecValTok{16}\NormalTok{,}\DecValTok{4}\NormalTok{,}
       \DecValTok{21}\NormalTok{,}\DecValTok{20}\NormalTok{,}\DecValTok{11}\NormalTok{,}\DecValTok{26}\NormalTok{,}\DecValTok{42}\NormalTok{,}\DecValTok{28}\NormalTok{,}\DecValTok{3}\NormalTok{,}\DecValTok{3}\NormalTok{,}\DecValTok{16}\NormalTok{,}\SpecialCharTok{{-}}\DecValTok{10}\NormalTok{,}
       \DecValTok{10}\NormalTok{,}\DecValTok{24}\NormalTok{,}\DecValTok{32}\NormalTok{,}\DecValTok{26}\NormalTok{,}\DecValTok{52}\NormalTok{,}\DecValTok{28}\NormalTok{,}\DecValTok{27}\NormalTok{,}\DecValTok{28}\NormalTok{,}\DecValTok{21}\NormalTok{,}\DecValTok{42}\NormalTok{)}

\NormalTok{n }\OtherTok{\textless{}{-}} \FunctionTok{length}\NormalTok{(y)}
\CommentTok{\# Cuatro columnas de la matriz de diseño}
\NormalTok{mu }\OtherTok{\textless{}{-}} \FunctionTok{rep}\NormalTok{(}\DecValTok{1}\NormalTok{,}\DecValTok{40}\NormalTok{)}
\NormalTok{alpha }\OtherTok{\textless{}{-}} \FunctionTok{c}\NormalTok{(}\FunctionTok{rep}\NormalTok{(}\DecValTok{1}\NormalTok{,}\DecValTok{10}\NormalTok{),}\FunctionTok{rep}\NormalTok{(}\DecValTok{0}\NormalTok{,}\DecValTok{10}\NormalTok{),}\FunctionTok{rep}\NormalTok{(}\DecValTok{0}\NormalTok{,}\DecValTok{10}\NormalTok{),}\FunctionTok{rep}\NormalTok{(}\DecValTok{1}\NormalTok{,}\DecValTok{10}\NormalTok{))}
\NormalTok{beta }\OtherTok{\textless{}{-}} \FunctionTok{c}\NormalTok{(}\FunctionTok{rep}\NormalTok{(}\DecValTok{0}\NormalTok{,}\DecValTok{10}\NormalTok{),}\FunctionTok{rep}\NormalTok{(}\DecValTok{1}\NormalTok{,}\DecValTok{10}\NormalTok{),}\FunctionTok{rep}\NormalTok{(}\DecValTok{1}\NormalTok{,}\DecValTok{10}\NormalTok{),}\FunctionTok{rep}\NormalTok{(}\DecValTok{0}\NormalTok{,}\DecValTok{10}\NormalTok{))}
\NormalTok{gamma }\OtherTok{\textless{}{-}} \FunctionTok{c}\NormalTok{(}\FunctionTok{rep}\NormalTok{(}\DecValTok{0}\NormalTok{,}\DecValTok{10}\NormalTok{),}\FunctionTok{rep}\NormalTok{(}\DecValTok{1}\NormalTok{,}\DecValTok{10}\NormalTok{),}\FunctionTok{rep}\NormalTok{(}\DecValTok{0}\NormalTok{,}\DecValTok{10}\NormalTok{),}\FunctionTok{rep}\NormalTok{(}\DecValTok{1}\NormalTok{,}\DecValTok{10}\NormalTok{))}
\end{Highlighting}
\end{Shaded}

\begin{Shaded}
\begin{Highlighting}[]
\NormalTok{cmod }\OtherTok{\textless{}{-}} \FunctionTok{lm}\NormalTok{(mu }\SpecialCharTok{\textasciitilde{}}\NormalTok{ alpha }\SpecialCharTok{+}\NormalTok{ beta }\SpecialCharTok{+}\NormalTok{ gamma)}
\NormalTok{cmod0 }\OtherTok{\textless{}{-}} \FunctionTok{lm}\NormalTok{(mu }\SpecialCharTok{\textasciitilde{}}\NormalTok{ gamma)}
\FunctionTok{anova}\NormalTok{(cmod0,cmod)}
\end{Highlighting}
\end{Shaded}

\begin{verbatim}
## Analysis of Variance Table
## 
## Model 1: mu ~ gamma
## Model 2: mu ~ alpha + beta + gamma
##   Res.Df        RSS Df  Sum of Sq  F Pr(>F)
## 1     38 5.6675e-29                        
## 2     37 5.5183e-29  1 1.4914e-30  1 0.3238
\end{verbatim}

\hypertarget{diseuxf1o-cross-over-simplificado-video}{%
\subsubsection{Diseño cross-over simplificado
(video)}\label{diseuxf1o-cross-over-simplificado-video}}

L ahipotesis principal es que alfa = beta. Como se trata de una unica
FPE, tenemos que q = 1 y se puede usar t de Student (otra vez todo a
mano). Mucho mejor como contraste de modelos.

En principio, el modelo general (g) tiene mu, alfa, beta y gamma.En
principio este modelo g tiene mu porque es la columna de los 1 del
principio, que siempre está en el lm.

El modelo de la hipotesis nula, si es cierta es que: La H0: alpha =
beta.

Si alfa es igual a beta la primera y la segunda columna se pueden unir
porque le parametro alfa e sigual a beta y solo hay uno. Si las unimos o
sumamos quedaria una columna de unos, que es igual a la mu. (Mirara la
Xr de diseño crossover simplificado).

La XR era:

mu = 1111 alfa = 1001 beta = 0110 gamma = 0101

alfa + beta = mu

La metriz de la hipotesis nula solo tendrá dso columnas entonces mu'=
1111 gamma = 0101

\begin{Shaded}
\begin{Highlighting}[]
\CommentTok{\# Modelo lineal con mu}
\NormalTok{g }\OtherTok{\textless{}{-}} \FunctionTok{lm}\NormalTok{(y }\SpecialCharTok{\textasciitilde{}}\NormalTok{ alpha }\SpecialCharTok{+}\NormalTok{ beta }\SpecialCharTok{+}\NormalTok{ gamma)}
\NormalTok{g0 }\OtherTok{\textless{}{-}} \FunctionTok{lm}\NormalTok{(y }\SpecialCharTok{\textasciitilde{}}\NormalTok{  gamma)}

\CommentTok{\# Test F:}
\FunctionTok{anova}\NormalTok{(g0, g)}
\end{Highlighting}
\end{Shaded}

\begin{verbatim}
## Analysis of Variance Table
## 
## Model 1: y ~ gamma
## Model 2: y ~ alpha + beta + gamma
##   Res.Df    RSS Df Sum of Sq      F  Pr(>F)  
## 1     38 6931.2                              
## 2     37 6147.9  1    783.23 4.7137 0.03641 *
## ---
## Signif. codes:  0 '***' 0.001 '**' 0.01 '*' 0.05 '.' 0.1 ' ' 1
\end{verbatim}

El resultado es una f de 4,71 el p valor es 0,036 y efectivamente el
resultado es el mismo que daba con la forma manual. Es decir la t de
estudent al cuadrado

\begin{Shaded}
\begin{Highlighting}[]
\CommentTok{\#Cálculo manual con G inversa}
\FunctionTok{library}\NormalTok{(MASS)}
\NormalTok{X }\OtherTok{\textless{}{-}} \FunctionTok{matrix}\NormalTok{(}\FunctionTok{c}\NormalTok{(mu, alpha, beta, gamma), }\AttributeTok{ncol=}\DecValTok{4}\NormalTok{)}
\NormalTok{r }\OtherTok{\textless{}{-}} \FunctionTok{qr}\NormalTok{(X)}\SpecialCharTok{$}\NormalTok{rank}
\NormalTok{XtX }\OtherTok{\textless{}{-}} \FunctionTok{crossprod}\NormalTok{(X) }\CommentTok{\# X crtapuesta de X}
\NormalTok{XtXinv }\OtherTok{\textless{}{-}} \FunctionTok{ginv}\NormalTok{(XtX) }\CommentTok{\# G inversa}
\NormalTok{param }\OtherTok{\textless{}{-}}\NormalTok{ XtXinv }\SpecialCharTok{\%*\%} \FunctionTok{crossprod}\NormalTok{(X, y)}
\NormalTok{a }\OtherTok{\textless{}{-}} \FunctionTok{c}\NormalTok{(}\DecValTok{0}\NormalTok{, }\DecValTok{1}\NormalTok{, }\SpecialCharTok{{-}}\DecValTok{1}\NormalTok{, }\DecValTok{0}\NormalTok{)}
\NormalTok{est }\OtherTok{\textless{}{-}} \FunctionTok{t}\NormalTok{(a) }\SpecialCharTok{\%*\%}\NormalTok{ param }\CommentTok{\#}
\NormalTok{residuos }\OtherTok{\textless{}{-}}\NormalTok{ y }\SpecialCharTok{{-}}\NormalTok{ X }\SpecialCharTok{\%*\%}\NormalTok{ param}
\NormalTok{MSE }\OtherTok{\textless{}{-}} \FunctionTok{sum}\NormalTok{(residuos}\SpecialCharTok{\^{}}\DecValTok{2}\NormalTok{)}\SpecialCharTok{/}\NormalTok{(n}\SpecialCharTok{{-}}\NormalTok{r) }\CommentTok{\# El mean square error, la estimación de sigma cuadrado}
\NormalTok{ee.est }\OtherTok{\textless{}{-}} \FunctionTok{sqrt}\NormalTok{(MSE }\SpecialCharTok{*} \FunctionTok{t}\NormalTok{(a) }\SpecialCharTok{\%*\%}\NormalTok{ XtXinv }\SpecialCharTok{\%*\%}\NormalTok{ a)}
\NormalTok{t.est }\OtherTok{\textless{}{-}}\NormalTok{ est}\SpecialCharTok{/}\NormalTok{ee.est }\CommentTok{\# divido el estimador entre el error estandar}
\NormalTok{t.est}\SpecialCharTok{\^{}}\DecValTok{2}
\end{Highlighting}
\end{Shaded}

\begin{verbatim}
##          [,1]
## [1,] 4.713676
\end{verbatim}

Rechazamos la igualdad entre los doas fármacos.

\hypertarget{resumen}{%
\section{Resumen:}\label{resumen}}

Mejor expresar los modelos aunque sean parametricos con el testF que con
t de Stdent

\end{document}
